\documentclass[12pt,a4paper]{article}

\usepackage{graphicx} % Required for inserting images.
\usepackage[margin=25mm]{geometry}
\usepackage[font=sf]{caption} % Changes font of captions.

\usepackage{amsmath}
\usepackage{amsfonts}
\usepackage{amssymb}
\usepackage{siunitx}
\usepackage{verbatim}
\usepackage{hyperref} % Required for inserting clickable links.


\title{Operating Systems and Distributed Systems\\Project 1 Part 1}
\author{Zhen Xuan}

\begin{document}
\maketitle

\section{Implementation}
To work concurrently, we use one enqueue thread to push image names. When finished,
it close the channel.\\
We set up several dequeue threads to resize images. Each dequeue thread constantly gets
an image and resize. When dequeue returns error code, channel is empty and thread will
stop.\\
When measuring latency, we support multiple enqueue threads and dequeue threads.
When a thread performs an enqueue or dequeue, we measure its time and sends to another
channel. Main thread will use this channel and collect timing result. Since totally 10000
enqueue and dequeue, main thread will also collect each operation 10000 times and then
return. Thus no dead lock will occur.
\section{Result}
We test our bounded queue's throughput and latency with different settings
of thread number and capacity.
\subsection{throughput}
Our result is as follows: Fig.~\ref{throughput}
\begin{figure}[htbp!]
    \begin{center}
        \includegraphics[width=\columnwidth]{./figure/throughput.png}
    \end{center}
    \caption{
        throughput with different (thread numbers, capacity)
    }
    \label{throughput}
\end{figure}\\
Here baseline is sequential resizing’s throughput. We use one enqueue thread and multiple
dequeue threads. We notice that the throughput gets larger as thread number grows. But
not change very much with more capacity. This shows that capacity is not the bottleneck.
More threads brings more concurrency, which will certainly increase the throughput.
\subsection{latency}
We measure enqueue and dequeue latency separately. Specifically, we test different ratio
of enqueue thread number and dequeue thread number. Our result is as follows: Fig.~\ref{enqueue},
Fig.~\ref{dequeue}
\begin{figure}[htbp!]
    \begin{center}
        \includegraphics[width=\columnwidth]{./figure/enqueue_latency_cdf.png}
    \end{center}
    \caption{lantency with different (enqueue, dequeue, capacity)}
    \label{enqueue}
\end{figure}
\begin{figure}[htbp!]
    \begin{center}
        \includegraphics[width=\columnwidth]{./figure/dequeue_latency_cdf.png}
    \end{center}
    \caption{lantency with different (enqueue, dequeue, capacity)}
    \label{dequeue}
\end{figure}
The reason we measure latency separately and with different ratio is that we find dequeue
latency is significantly larger than enqueue latency. (Enqueue latency is usually several
microseconds, while dequeue is several thousands.)\\
This suggests that enqueue is not the bottleneck as well as capacity, which also confirms
what we find in throughput.\\
With larger ratio of enqueue over dequeue, we see that dequeue latency drops, which
strongly proves that enqueue is the bottleneck.
\end{document}